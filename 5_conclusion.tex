\begin{conclusion}
Podařilo se mi navrhnout a implementovat systém, který dokáže spolehlivě vyhledávat a shlukovat české textové dokumenty vyšetřovatelů. Je to zatím jen první část celkového navrženého komplexního řešení, které by vyšetřovatelé potřebovali. K dalším částem bude potřeba integrovat další, ještě pokročilejší metody zpracování textu.

V práci jsem využil pouze open source technologie, především aplikace Solr a Carrot2, do kterých jsem implementoval nové nástroje pro zpracování českých textů. Speciálně jsem se zaměřil na nástroje pro získání kořenů slov, takzvané stemmery. Kvality těchto nástrojů jsem otestoval a vybral ty nejvhodnější pro účely vyhledávání v textu. 

Mnou navržený český stemmer (respektive kombinace stemmerů) je dle mých testů o 10\% procent úspěšnější než původní stemmer přednastavený v Solru. K tomu jsem navíc implementoval i další stemmery, z nichž jeden má zas vysokou úspěšnost tvorby správných lingvistických kořenů, což může být využito v dalších fázích implementace budoucího řešení zpracování textu.

K dosažení všech definovaných cílů jsem vytvořil několik přidružených aplikací, z nichž jednou je  \emph{Crawler}, aplikace která se k Solru připojuje a která má za úkol přidávání a odebírání dokumentů z indexu. Aplikaci se mi podařilo navrhnout plně modulární a tedy i do budoucna rozšiřitelnou.

Solru jsem také postavil webový frontend pro uživatelské vyhledávání v dokumentech, postavený na moderních technologiích jako HTML5, CSS3 či Angular.js. Frontend je navržený s důrazem na intuitivnost, což bylo ověřeno uživatelským testováním.

Výsledný systém jsem se snažil koncipovat tak, aby mohl být dále rozšiřován o další moduly či aplikace tak, aby v dalších fázích projektu mohlo být na tuto implementovanou část navázáno. Tomu jsem podřídil výběr technologií i dalších prostředků vývoje.

Po celou dobu vývoje jsem své kroky korigoval se zadavatelem projektu, abych co nejvíce uspokojil jeho potřeby. Ale i tak se již objevují nové nápady, jak aplikaci vyhledávání ještě dále vylepšit. Zadavatel by například uvítal možnost vyhledat dokumenty i z několika kolekcí (myšleno indexů) naráz. Technicky by nový požadavek mohl být například splněn úpravou frontendu, který by dotaz rozeslal do vybraných kolekcí a výsledky sloučil. Tato možnost poměrně jednoduchého řešení dodatečných požadavků pouze dokazuje úspěšný rozšiřitelný návrh řešení.
\end{conclusion}