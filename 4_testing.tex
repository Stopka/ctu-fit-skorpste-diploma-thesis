\chapter{Testování} \label{testing}
\section{Test stemmerů}
\subsection{Správnost kořenů}
Tato část je věnována testování kvality jednotlivých implementovaných stemmerů. Helebrand pro účely testování svého stemmeru vytvořil seznamy slov, ve kterých ručně označil správné kořeny. V seznamu jsou u slov uvedeny dva kořeny, jeden lingvisticky naprosto správný, druhý méně správný, ale uznatelný tvar.

Seznamy jsou připraveny dva. Jeden obsahuje slova vybraná z reálného článku a umožní otestovat nasazení algoritmu v reálném prostředí. Druhý seznam slov je připravený speciálně pro otestování správné funkce stemmerů pro různé tvary slov včetně méně obvyklých tvarů a vyjímek. Obsažená slova jsou uvedena ve všech pádech a číslech. Slova jsou v seznamech tříděna do souborů podle slovních druhů, z~výsledků tedy bude možné odvodit úspěšnost získávání kořene slova jednotlivých slovních druhů.

V jazyce Ruby jsem naprogramoval krátký jednoúčelový skript, který na základě parametrů načte příslušný seznam slov a zparsuje výchozí tvary slov a jejich označené kořeny. Jednotlivá slova pak nechá převést na jejich kořeny analyzerem s příslušným stemmerem tak, že je přímo pošle na REST rozhraní Solru. Získaný kořen pak porovná s~označenými kořeny v~seznamu. Průběžné výsledky zapíše do výstupního souboru, aby bylo možné provádět hlubší analýzy. Nakonec skript zobrazí a  do výstupního souboru zapíše číselné hodnoty, které odpovídají celkovému počtu slov a~počtu slov, která byla stemována na žádaný tvar. Tímto způsobem lze vyjádřit procentuální úspěšnost algoritmu.

Pomocí tohoto skriptu jsem otestoval všechny implementované české stemmery. Výsledky jsou uvedeny v~tabulce \ref{tab:test_stm}. Z~ní také můžeme vyčíst, že v~tomto testu si vede nejhůře původní stemmer. Nejlépe pak dopadl Helebrand stemmer. Výsledky testu rovněž prokázaly, že slovníkový stemmer Hunspell není celkově příliš účinný, ale má velmi dobré výsledky v kategorii sloves --- pravděpodobně proto, že jde spíše o~lematizer, který vrací slova v~základním tvaru, takže se utvořený tvar neshoduje s~označným kořenem. Malá účinnost je zároveň důsledkem nízké kvality českého slovníku, neboť je třeba brát v potaz, že v době jejich vzniku byly určeny ke kontrole pravopisu.

V rámci další analýzy výsledků algoritmů jsem procházel výstupní soubory s vytvořenými kořeny a zjistil jsem, že nenalezne-li Hunspell slovo ve slovníku, vrací původní tvar slova. Napadlo mě tedy zkusit spojit dva stemmery za sebe. Slovo by se nejdříve předložilo Hunspell stemmeru. Kdyby nebyl úspěšný, slovo by zůstalo v původním tvaru a je možné jej předložit stemmeru dalšímu. Když bude úspěšný, vrátí slovo v základním tvaru podle slovníku a další stemmer v pořadí bude mít jednodušší práci s ořezáním slova na jeho kořen. V obou případech by měl být výsledkem kořen slova použitelný pro indexaci. V konfiguraci Solru jsem tedy vytvořil další analyzer obsahující dva stemmery. Jako druhý stemmer po slovníkovém jsem zvolil ten, který měl v testu nejlepší výsledek --- Helebrandův.

Výsledný analyzer jsem otestoval stejným způsobem jako předešlé a~zanesl výsledek do tabulky \ref{tab:test_stm} pod názvem \emph{text\_cz\_hh}. Analyzer dosáhl rekordní úspěšnosti 70,98\%. Obě metody se dle očekávání doplňují. Slovník přispívá vysokou přesností, druhý stemmer možností univerzálního použití pro kterékoliv slovo.

\begin{table}
\begin{center}
\begin{tabular}{|l|r|r|r|r|r|r|r|r|}
\hline
(\%) & \multicolumn{2}{|c|}{\textbf{Podst. jm.}} & \multicolumn{2}{|c|}{\textbf{Příd. jm.}} & \multicolumn{2}{|c|}{\textbf{Slovesa}} & \multicolumn{2}{|c|}{\textbf{Vše}} \\ \hline
\textbf{Stemmery} & \textbf{Čl.} & \textbf{Tst.} & \textbf{Čl.} & \textbf{Tst.} & \textbf{Čl.} & \textbf{Tst.} & \textbf{Čl.} & \textbf{Tst.} \\ \hline
cz & 51,68 & 3,07 & 39,69 & 4,15 & 7,58 & 2,44 & 38,50 & 3,20 \\ \hline
cz\_light & 60,40 & 3,80 & 46,56 & 2,63 & 11,36 & 2,64 & 45,63 & 3,40 \\ \hline
cz\_agressive & 55,03 & 48,53 & 52,67 & 23,37 & 3,03 & 0,00 & 42,25 & 36,61 \\ \hline
cz\_helebrand & 53,69 & 58,47 & 49,62 & 47,16 & 31,82 & 38,62 & 47,59 & 53,39 \\ \hline
cz\_hunspell & 35,91 & 0,00 & 1,53 & 0,00 & 70,45 & 90,04 & 38,04 & 12,55 \\ \hline
\hline
cz\_hh & 56,04 & 74,42 & 54,20 & 66,39 & 39,39 & 61,59 & 51,69 & 70,98 \\ \hline
\end{tabular}
\end{center}
\caption{Výsledek testu správnosti kořenů}
\label{tab:test_stm}
\end{table}

\subsection{Shodnost kořenů}
Po další důkladné analýze výsledků předchozího testu jsem zjistil, že test nevypovídá dostatečně o vhodnosti nasazení stemmeru pro indexaci a~vyhledávání. Pro nejlepší výsledky je nutné, aby stemmer tvořil pokud možno stejný kořen pro všechny tvary jednoho slova, což se v~mnoha případech nedělo.

Připravil jsem tedy test nový. Odvodil jsem nový seznam slov, kde na řádku jsou vždy uvedeny různé tvary téhož slova. Upravil jsem také testovací algoritmus, který tentokrát postupně nechá všechny tvary jednoho slova  zpracovat analyzerem Solru a porovná vytvořené kořeny mezi sebou. Poté spočítá, kolik procent kořenů se shoduje. Čím více slov se tedy zpracuje do stejného kořene, tím vyšší číslo a tím lepší podal algoritmus výsledek na daném slově. Celkovou úspěšnost algoritmu odvozuji ze dvou čísel a to z průměrné úspěšnosti, získané aritmetickým průměrem úspěšnosti na jednotlivých slovech a úspěšnosti, počítané jako procento slov, u nichž se všechny tvary převedly na shodný tvar (jinými slovy na kolika procentech slov měl algoritmus stoprocentní úspěšnost).

Výsledky testu jsem můžeme pozorovat v tabulce \ref{tab:test_eql}. První sloupec udává pro daný slovní druh průměrný počet nalezených shodných kořenů, druhý pro kolik slov byl nalezen shodný kořen pro všechny jejich tvary. Tento test ukazuje, že nejlepších výsledků dosahuje slovníkový Hunspell stemmer. Druhým nejlepším je původní stemmer zabudovaný v Solru.

Oba stemmery předstihuje pouze kombinace Hunspell a Helebrnd stemmeru, vytvořená po minulém testu, což potvrzuje funkčnost navrženého principu kombinování. Protože jsou výsledky tohoto testu odlišné než výsledky testu předchozího, vytvořil jsem nový analyzer s použitím nejúspěšnějších stemmerů v nové testu --- Hunspell a původní stemmer. Nový kombinovaný analyzer jsem také podrobil testu a výsledky zanesl do tabulky pod jménem \emph{text\_cz\_hx}. Nově vytvořený stemmer dosahuje nejlepších výsledků a~zdá se tak být pro vyhledávání nejvhodnějším kandidátem.

V konfiguraci Solru jsem na základě výsledků testu nastavil pro pole \emph{text} datový typ \emph{text\_cz\_hx} sestávající z~po sobě jdoucích \emph{solr.Hunspell} stemmeru a~\emph{solr.CzechStemmer}.

\begin{table}
\begin{center}
\begin{tabular}{|l|r|r|r|r|r|r|r|r|}
\hline
(\%) & \multicolumn{2}{|c|}{\textbf{Podst. jm.}} & \multicolumn{2}{|c|}{\textbf{Příd. jm.}} & \multicolumn{2}{|c|}{\textbf{Slovesa}} & \multicolumn{2}{|c|}{\textbf{Vše}} \\ \hline
\textbf{Stemmery} & \textbf{Pr.} & \textbf{Úsp.} & \textbf{Pr.} & \textbf{Úsp.} & \textbf{Pr.} & \textbf{Úsp.} & \textbf{Pr.} & \textbf{Úsp.} \\ \hline
cz & 98,38 & 92,00 & 94,60 & 40,00 & 24,95 & 1,82 & 85,09 & 62,22 \\ \hline
cz\_light & 87,40 & 38,00 & 81,38 & 3,64 & 23,50 & 1,52 & 75,08 & 21,11 \\ \hline
cz\_agressive & 89,57 & 54,00 & 91,75 & 31,64 & 39,35 & 1,82 & 81,81 & 37,78 \\ \hline
cz\_helebrand & 77,91 & 33,27 & 59,92 & 3,27 & 52,27 & 5,15 & 68,64 & 18,48 \\ \hline
cz\_hunspell & 86,66 & 44,18 & 84,10 & 32,00 & 94,39 & 66,67 & 87,24 & 44,55 \\ \hline
\hline
cz\_hh & 90,44 & 51,09 & 90,00 & 32,00 & 94,39 & 66,67 & 90,97 & 48,38 \\ \hline
cz\_hx & 97,05 & 89,82 & 94,60 & 40,00 & 94,39 & 66,67 & 95,92 & 72,12 \\ \hline
\end{tabular}
\end{center}
\caption{Výsledek testu shody kořenů}
\label{tab:test_eql}
\end{table}

\section{JUnit}
Všechny zásadní celky kódu jsou pokryty testy, které kontrolují integritu komponent pomocí \emph{JUnit}. Testy budou užitečné při případném dalším rozšiřování. Pokud dojde budoucí úpravou k~nechtěné zásadní změně funkčnosti existujících komponent, měly by to testy odhalit.

V~aplikaci Crawler jsou takto pokryty všechny procesorové komponenty a~crawlery. V~rozšiřující Solr knihovně \emph{Analyzery.jar} jsou pokryty všechny tři nové stemmery i~s~jejich továrními třídami.

Všechny napsané testy aktuálně hlásí stoprocentní úspěšnost.

\section{Test použitelnosti}
V tomto testu byla aplikace předložena skutečným uživatelům, aby otestovali navržené uživatelské rozhraní. Ověřím tak intuitivnost a funkčnost návrhu i imeplementace.

\subsection{Výběr uživatelů}
Testovací uživatele jsem volil především podle kritéria zkušenosti s prací s internetovými vyhledávači. Vybral jsem uživatele, kteří se běžně pohybují na internetu, umějí ovládat internetový prohlížeč, mají zažité zvyklosti z tohoto prostředí a běžně vyhledávají informace pomocí internetových vyhledávačů. Toto kritérium by mělo přesně vymezit skupinu uživatelů, pro kterou bylo uživatelské rozhraní určeno.

Uživatele jsem se rozhodl vybírat z okruhu svých přátel a to z několika důvodů. Jednak u nich mám představu o úrovni jejich znalostí a schopností, tudíž si mohu zvolit takové testovací uživatele, kteří spadají do výše zmíněné cílové skupiny. Takoví uživatelé ke mně budou během provádění testu také více otevření, tudíž se dozvím s menším úsilím více detailů o problémech, se kterými se potýkali.

\subsection{Charakteristika uživatelů}
Uživatelka Karolína je studentkou Filozofické fakulty Univerzity Karlovy. Počítač využívá k tvorbě školních prací, k práci s internetem, komunikaci i sledování multimédií. Je schopná na počítači provést i některé složitější konfigurační úkony, její uživatelské schopnosti tak hodnotím jako vysoké. Na internetu se pohybuje každodenně, často hledá informace skrze internetové služby, sleduje video portály, navštěvuje sociální sítě. Taktéž její zkušenost s internet hodnotím jako vysokou.

Uživatelka Petra používá počítač pro prohlížení a zálohu fotografií, práci s dokumenty a práci na internetu. Konfigurační úkony jsou pro ni velkou neznámou. Na internetu využívá sociální sítě, e-mailové služby, zpravodajské weby a vyhledávače. Web používá i k prodeji a propagaci zboží. Její počítačové schopnosti jsem tedy vyhodnotil spíše jako průměrné, internetové jako vysoké.

Uživatel Miroslav je absolventem Strojní fakulty ČVUT, jeho počítačové zkušenosti považuji za průměrné. Počítač používá pouze pro psaní dokumentů a k práci na internetu. Internet používá převážně k e-mailové komunikaci. Informace na internetu vyhledává v porovnání s předchozími uživatelkami méně často.

Anna je studentkou Ústavu translatologie na Filozofické fakultě Univerzity Karlovy. Počítač používá pro plnění školních povinností i pro zábavu. Na internetu používá sociální sítě, e-mailové služby, zpravodajské weby i vyhledávání. Ve vyhledávání na internetu je velmi pokročilá, denně při překladu vyhledává správná užití frází pomocí pokročilých technik dotazování internetových vyhledávačů.

Posledním uživatelem, který se účastnil testu, je Vladimír. Vladimír je studentem Technické univerzity v Liberci. V testu je nejpokročilejším uživatelem. Ve volném čase se zabývá programováním webových stránek i složitějších aplikací.

Základní orientační charakteristiku vybraných uživatelů znázorňuje tabulka \ref{tab:test_usrs}.

\begin{table}
\begin{center}
\begin{tabular}{|l|l|l|l|l|l|}
\hline
\textbf{Jméno} & \textbf{Věk} & \textbf{Pohlaví} & \textbf{Vzdělání} & \textbf{PC gram.} & \textbf{WWW gram.} \\ \hline
Karolína & 21 & Žena & Maturita & Vysoká & Vysoká \\ \hline
Petra & 46 & Žena & Maturita & Průměrná & Vysoká \\ \hline
Ing. Miroslav & 54 & Muž & Vysokoškolské & Průměrná & Průměrná \\ \hline
Anna & 21 & Žena & Maturita & Vysoká & Vysoká \\ \hline
Vladimír & 20 & Muž & Maturita & Expertní & Expertní \\ \hline
\end{tabular}
\end{center}
\caption{Testovací uživatelé}
\label{tab:test_usrs}
\end{table}

\subsection{Průběh testu}
Testovacím uživatelům jsem stručně představil aplikaci, seznámil jsem je s účelem aplikace, v bodech jsem vyjmenoval možnosti, které nabízí, a vysvětlil jsem jim základní terminologii. Z celého průběh testu jsem se svolením uživatelů pořídil videonahrávku, abych mohl test zpětně analyzovat a vyvodit závěry. Test jsem prováděl s každým uživatelem zvlášť.

Do systému jsem pro účely testu nahrál data --- do jednoho indexu dokumenty s uloženými články z technického zpravodajského webu, do druhého anonymizované příklady policejních dokumentů. Uživatele jsem nechal několik minut zkoumat prostředí, aby se sami zběžně seznámili s funkcemi. Poté jsem uživatelům zadal následující úkoly a instruoval je, aby své kroky, nahlas komentovali. Během plnění úkolu jsem nezasahoval, pouze jsem pasivně pozoroval jejich počínání.

\begin{enumerate}
\item Vyhledejte dokumenty týkající se firmy \uv{Microsoft}.
\item Zjistěte, jaké shluky se ve výsledcích vyskytují.
\item Zjistěte, v jakých shlucích se nachází první výsledek.
\item \label{task_clusterfilter}Odfiltrujte pouze výsledky ve shlucích \uv{Windows 10} a \uv{Google}.
\item Zobrazte z novu všechny výsledky hledání.
\item Otevřete první dokument ve výsledcích.
\item Otevřete dokument, který je otevřenému dokumentu nejpodobnější.
\item Vyhledejte v druhé kolekci dokumenty, které obsahují spojení \uv{úřední záznam}.
\item \label{task_engine}Změňte mechanismus shlukování.
\item Zjistěte, kdy byly nalezené dokumenty naposledy upraveny.
\end{enumerate}

Úkoly byly navrženy tak, aby prověřily především klíčové aspekty rozhraní, jako například zda uživatel bude vědět, jak se vrátit z detailu dokumentu, zda uživatel pochopí funkci filtrování podle shluků a zda pochopí možnost přepínání kolekcí či shlukovacích algoritmů.

Po dokončení testu jsem s každým uživatelem ještě diskutoval o návrzích na vylepšení a dotázal jsem se na jejich porozumění funkcím jednotlivých detailů prvků rozhraní. Zaměřil jsem se na následující body.

\begin{itemize}
\item Jestli vědí, co vyjadřuje číslo u každého shluku.
\item Jestli bezpečně rozeznají otevřené shluky.
\item Jestli dokáží zjistit, do jakých shluků dokument patří.
\item Jestli vědí, jaké výsledky se zobrazují ve výchozím stavu.
\item Jestli rozumí metadatům dokumentu.
\end{itemize}

Po každém testu jsem vždy rovnou navrhl a implementoval řešení nalezených kritických problémů, abych v dalších testech ihned viděl, zda má opatření účinek.

\subsection{Výsledky testu}
\subsubsection{Karolína}
Jako první jsem test provedl s uživatelkou Karolínou. Uživatelka neměla problém s rozeznáním funkcí uživatelského rozhraní. Největší problém nastal v bodě \ref{task_clusterfilter}, kde měla vybrat výsledky jen z daných shluků. Uživatelku nenapadlo kliknout na více shluků a tím jich více vybrat. V následné diskusi jsem zjistil, že se tak stalo kvůli nedostatečným zvýrazněním otevřených shluků. Otevřený shluk nyní změní pouze přidruženou ikonu složky z běžné na otevřenou, což uživatelka nepostřehla, nebo nebrala v úvahu. Zároveň jí zmátlo barevné označení odkazu po kliknutí. Prohlížeč Chrome totiž neodebírá správně \emph{hover} efekt odkazu po odjetí myši z odkazu.

Na základě tohoto poznatku jsem dodatečně provedl drobnou úpravu frontendu. Řádek s otevřeným shlukem se označí nejen ikonou otevřeného adresáře, ale i změnou barvy. Tím by mělo dojít k dostatečnému zvýraznění otevřených shluků.

\subsubsection{Petra}
Následně jsem test provedl s uživatelkou Petrou. Ta se v prostředí také rychle zorientovala. Problém nastal opět u úkolu \ref{task_clusterfilter}. Aktuální vizualizace otevřeného shluku stále dostatečně nenapovídá, že lze filtrovat výsledky podle více shluků.

Další problém nastal v bodě \ref{task_engine}. Možnost přepnout shlukovací algoritmus hledala v prostoru bloku s nalezenými shluky a vůbec ji nenapadlo otevřít rozšířenou nabídku vyhledávání. Nabídku jsem dle zvyklostí z jiných vyhledávačů umístil vpravo od vyhledávacího pole. Rozšířenou nabídku uživatelka nepoužívá ani v běžných internetových vyhledávačích. Tento úkon naštěstí není nutný pro běžné vyhledávání, proto je nabídka běžně schovaná a proto také není velkým problémem, když ji uživatel hned nenajde. 

Diskuzí jsme nakonec dospěli ještě k jedné nesrovnalosti. Při vyhledávání je každý výsledek zobrazen se třemi úryvky textu, ve kterých jsou zvýrazněna vyhledávaná slova. Každý úryvek je oddělen vizuálně novým řádkem tabulky a to se zdálo být matoucí.

Na základě výsledků tohoto testu jsem opět vyvodil drobné změny v návrhu UI. Znovu jsem opravil zvýraznění otevřených shluků v seznamu a nahradil původní ikonu otevřeného či zavřeného adresáře ikonou zaškrtnutého či nezaškrtnutého formulářového políčka \emph{checkbox}. Uživatelé jsou zvyklí z formulářů na internetu, že zaškrtávací pole ve tvaru čtverce znamenají možnost vybrat více položek, tato úprava by by jim tedy měla napovědět i v tomto případě.

Dále jsem se pokusil řešit problém s úryvky textu ve výsledcích. Existují dvě možnosti. Buď se zobrazí jen jeden delší úryvek, nebo se změní způsob zobrazení úryvků tak, aby byly méně matoucí. Rozhodl jsem se pro druhou možnost. Nadále se budou zobrazovat tři úryvky, neboť mi připadá výhodné zobrazit více relevantního textu s hledanými slovy, což pak usnadní hledání požadovaného dokumentu. Změní se však způsob zobrazení úryvků. Úryvky se nebudou zobrazovat ve třech samostatných řádcích, ale budou v jednom bloku odděleny obyčejnou textovou výpustkou.

\subsection{Ing. Miroslav}
Miroslav měl s použitím softwaru největší potíže. Největší problém představovalo najít schovanou nabídku přepínání shlukovacího algoritmu. Hledal ji spíše v místě nabídky kolekcí, ale nakonec ji úspěšně našel. Navíc očekával, že změna algoritmu proběhne ihned při změně ve vstupním poli, a byl zmaten, když se tak nestalo.

Miroslav si také nebyl jistý funkcí filtrování podle shluků. Ihned ale pochopil, že může zaškrtnout i více položek, takže předchozí opravná opatření se zdají být úspěšná. Nebyl si však jistý jak se výsledek filtrování projevil. Nepostřehl, že se mezi výsledky zobrazují jen některé dokumenty. Rozhodl jsem se tedy provést ještě jednu drobnou změnu: u každé položky se zvýrazní barevně shluky, podle kterých se filtruje.

\subsection{Anna}
Anna zvládla všechny úkoly bez zjevného zaváhání. V následné diskuzi o proběhlém testu se pouze přiznala, že hned nevěděla, jak otevřít detail dokumentu. Zkoušela kliknout nejdříve na text dokumentu ve výsledcích, než si všimla, že je v hlavičce odkaz.

Zvažoval jsem, že bych jako odkaz na detail dokumentu označil celý box patřící položce výsledku. Tuto možnost jsem však nakonec zavrhl. Uživatelé by například mohli mít potíže s kopírováním textů a podobně.

\subsection{Vladimír}
Poslední test jsem provedl s uživatelem Vladimírem, přičemž s jeho zkušenostmi by se tento test dal označit spíše za expertní test. Uživatel neměl s použitím aplikace žádný větší problém. Jeho jedinou drobnou připomínkou bylo, že na stránce s detailem dokumentu hledal tlačítko pro návrat, než si uvědomil, že může použít funkce prohlížeče.

\section{Akceptační test}
Posledním provedeným testem byl akceptační test. Prošel jsem jednotlivé definované požadavky na systém a~ověřil jejich splnění. Všechny požadavky, které jsem zařadil do první fáze projektu a~kterým se věnuje tato práce jsou splněny. Ostatní požadavky jsou odloženy, jelikož nebylo v~mých silách implementovat celé tak rozsáhlé zadání. Detailní rozpis je uveden v~tabulce \ref{tab:test_accept}.
\begin{table}
\begin{center}
\begin{tabular}{|l|l|p{6cm}|}
\hline
\multicolumn{3}{|c|}{\textbf{\ref{req_0} Funkční požadavky}} \\ \hline
\multicolumn{2}{|l|}{\textbf{Požadavek}} & \textbf{Stav} \\ \hline
\ref{req_00} & Indexace textových souborů & Splněno\\ \hline 
\ref{req_01} & Rozšíření o~další index dokumentů & Splněno\\ \hline 
\ref{req_02} & Ukládání celého obsahu dokumentů & Splněno\\ \hline 
\ref{req_03} & Zpracovávání textu v~českém jazyce & Splněno\\ \hline 
\ref{req_04} & Fulltextové vyhledávání v~dokumentech & Splněno\\ \hline 
\ref{req_040} & Tokenizace českého textu & Splněno\\ \hline 
\ref{req_041} & Stemizace českého textu & Splněno\\ \hline 
\ref{req_042} & Zobrazení obsahu nalezených dokumentů & Splněno\\ \hline 
\ref{req_043} & Zvýraznění hledaných slov & Splněno\\ \hline 
\ref{req_05} & Zobrazení podobných dokumentů & Splněno\\ \hline 
\ref{req_050} & Počítání podobnosti dokumentů & Splněno\\ \hline 
\ref{req_06} & Shlukování výsledků hledání & Splněno\\ \hline 
\ref{req_060} & Shlukování do automatických kategorií & Splněno\\ \hline 
\ref{req_061} & Filtrování výsledků podle shluků & Splněno\\ \hline 
\ref{req_07} & Rozpoznávání jmenných entit & Odloženo do dalších fází\\ \hline 
\ref{req_070} & Rozpoznávání větných členů & Odloženo do dalších fází\\ \hline 
\ref{req_071} & Rozpoznávání slovních druhů & Odloženo do dalších fází\\ \hline 
\ref{req_072} & Porovnávání s~jmennými databázemi & Odloženo do dalších fází\\ \hline 
\ref{req_08} & Extrakce vazeb mezi entitami & Odloženo do dalších fází\\ \hline 
\hline
\multicolumn{3}{|c|}{\textbf{\ref{req_1} Nefunkční požadavky}} \\ \hline
\multicolumn{2}{|l|}{\textbf{Požadavek}} & \textbf{Stav} \\ \hline
\ref{req_10} & Čistě open source technologie & Splněno\\ \hline 
\ref{req_11} & Architektura klient-server jako webová služba & Splněno\\ \hline
\ref{req_12} & Nativně nasaditelný pod Windows & Splněno\\ \hline
\ref{req_13} & Rozšiřitelnost & Splněno\\ \hline
\end{tabular}
\end{center}
\caption{Výsledek akceptačního testu}
\label{tab:test_accept}
\end{table}