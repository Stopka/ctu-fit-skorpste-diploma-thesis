\begin{introduction}
Policie České republiky uchovává velké množství dokumentů. Jedná se o~výpovědi, oznámení a~stručné zprávy týkající se různých případů v~čistě textové podobě, bez jakékoliv pevné struktury či atributů, psané v~českém spisovném jazyce. Z~těchto dokumentů je potřeba seskládat či vytěžit jakékoliv užitečné informace, které by vedly k~dopadení pachatelů.

Nástrojů v~oblasti získávání různých druhů dat z~textu a~jejich vizualizace existuje celá řada, mezi nejpoužívanější patří Rapid Miner, R či GATE. Tyto programy většinou disponují nástroji pro zpracování anglického textu, popřípadě pro zpracování textu v~jiném světovém jazyce. Horší je situace s~nástroji pro zpracování dokumentů v~českém jazyce. Existuje několik komerčních řešení, avšak žádný jednoduše použitelný open source program.

Z těchto důvodů se zadavatel práce obrátil s~problémem na Fakultu informačních technologií ČVUT. Na následujících stránkách této práce se pokusím navrhnout a~implementovat nástroje pro zpracování českého textu a~aplikaci, která pomůže vyšetřovatelům v~jejich práci.
\end{introduction}