\chapter{Uživatelská příručka}
\section{Požadavky}
Systém ke svému fungování potřebuje následující softwarové prerekvizity. Aplikace může fungovat i~na jiných verzích, na těchto zde napsaných byla aplikace testována.
\begin{itemize}
\item Windows 7/8; Linux
\item Java runtime 1.8 (na počítači \emph{search~server} a~\emph{file~client}, viz kapitola\ref{arch})
\item Internetový prohlížeč, např. Chrome~42/Firefox~37 (na počítači \emph{search~client}, viz kapitola \ref{arch})
\item Apache~Maven~3 (pouze pro kompilaci)
\item IntelliJ~Idea (doporučeno, pouze pro kompilaci)
\end{itemize}

\section{Instalace}
\begin{enumerate}
\item Zkopírujte adresář s~aplikací \emph{searcher} do adresáře cílového počítače (\emph{search~server} - viz kapitola \ref{arch}).
\item Zkopírujte adresář s~aplikací \emph{crawler} do adresáře cílového počítače (\emph{file~client} - viz kapitola \ref{arch})
\end{enumerate}

\section{Kompilace}
Kompilaci mají všechny součásti velmi podobnou, proto zde uvedu jednotný návod.
\begin{enumerate}
\item V~adresáři zdrojových souborů spusťte program Maven příkazem \\ \verb|maven clean build|.
\item Maven automaticky stáhne potřebné knihovny a~sestaví knihovnu, webový archiv či java aplikaci. Hotový archiv naleznete v~adresáři \emph{/target/} příslušné aplikace.
\end{enumerate} 

\subsection{Sestavení}
\begin{itemize}
\item Aplikace \emph{crawler} je samostatnou aplikací běžící v~Java runtime. Všechny závislosti se při kompilaci přibalí do výsledného aplikačního archivu \emph{crawler-jar-with-dependencies.jar}
\item Knihovnu \emph{Analyzery} nainstalujete do solru zkopírováním výsledného souboru\emph{Analyzery.jar} do adresáře \emph{/searcher/solr/lib/} serverové aplikace.
\item \emph{Frontend} nainstalujete do solru zkopírováním výsledného souboru \emph{frontend.war} do adresáře \emph{/searcher/webapps/}.
\end{itemize}

\section{Searcher}
\subsection{Spuštění}
Aplikaci spustíte zavoláním skriptu \emph{run.bat} na Windows, respektive \emph{run.sh} na Linuxu. Po každé úpravě konfigurace je nutné aplikaci restartovat.

\subsection{Konfigurace}
\subsubsection{Změna portu}
Síťový port nastavíte v~souboru \emph{/searcher/etc/jetty.xml} změnou hodnoty označené textem \emph{[PORT]} v~následujícím úryvku. Ve výchozím stavu je port nastaven na \emph{8080}.
\begin{verbatim}
<Call name="addConnector">
  <Arg>
    <New class="org.eclipse.jetty.server.bio.SocketConnector">
      <Set name="port">
        <SystemProperty name="jetty.port" default="[PORT]"/>
      </Set>
    </New>
  </Arg>
</Call>
\end{verbatim}

\subsubsection{Přidání indexu}
Vytvořte kopii adresáře \emph{/searcher/solr/collection1} pro český index, nebo \emph{/searcher/solr/en\_collection1} pro anglický. Nový adresář vhodně pojmenujte, například \emph{/searcher/solr/collection2}. Upravte jméno kolekce změnou hodnoty \verb|name=collection2| v~souboru \emph{/searcher/solr/collection2/core.properties}. Jméno kolekce musí být unikátní.

\subsubsection{Změna stemmeru}
Nahraďte \emph{[ANALYZER]} v~úryvku souboru \\ \emph{/searcher/solr/[KOLEKCE]/conf/schema.xml} jménem analyzéru s~požadovaným stemmerem.
\begin{verbatim}
<field name="text" type="[ANALYZER]" indexed="true" 
       stored="true" termVectors="true" />
\end{verbatim}
Přípustné hodnoty jsou \emph{text\_cz}, \emph{text\_cz\_light}, \emph{text\_cz\_agressive}, \\ \emph{text\_cz\_helebrand}, \emph{text\_cz\_hunspell}, \emph{text\_cz\_hh}, \emph{text\_cz\_hx}. Výchozí hodnota je \emph{text\_cz\_hx}. Po změně je nutné přeindexovat všechny soubory.

\subsubsection{Změna výchozího shlukovacího algoritmu}
Nahraďte \emph{[ENGINE]} v~úryvku souboru \\ \emph{/searcher/solr/[KOLEKCE]/conf/solrconfig.xml} jménem požadovaného shlukovacího enginu. Slovo \emph{[KOLEKCE]} v cestě nahraďte jménem adresáře s kolekcí, jejíž chování chcete upravit.
\begin{verbatim}
<requestHandler name="/searcher" startup="lazy"
                enable="${solr.clustering.enabled:false}"
                class="solr.SearchHandler">
  <lst name="defaults">
    <str  name="clustering.engine">[ENGINE]</str>
  </lst>
</requestHandler>
\end{verbatim}
Přípustné hodnoty jsou \emph{lingo} (výchozí), \emph{stc}, \emph{kmeans}.

\subsubsection{Přidání stop slov}
Do seznamu stop slov lze přidat nové slovo vytvořením nového řádku se slovem v konfiguračním souboru \\ \emph{/searcher/solr/[KOLEKCE]/conf/lang/stopwords\_[JAZYK].xml}. Slovo \emph{[KOLEKCE]} v cestě nahraďte jménem adresáře s kolekcí, jejíž chování chcete upravit. Slovo \emph{[JAZYK]} pak mezinárodním dvoupísmeným kódem jazyka kolekce.

\section{Crawler}
Crawler se spuští zavoláním skriptu \emph{run.bat} na Windows, respektive \emph{run.sh} na Linuxu. V~rámci jednoho běhu se provedou dvě fáze: \emph{cleaning} a~\emph{indexing}. \emph{Indexing} fáze prochází adresáře, hledá nové soubory a~všechny soubory zanese do indexu. Existuje-li již soubor v~indexu, aktualizuje jej. \emph{Cleaning} fáze projde všechny dokumenty v~indexu a~pokud již soubor neexistuje v~adresáři, z~indexu jej smaže. 

Veškerá konfigurace se provádí skrze konfigurační soubor či vstupní parametry. Ve výchozím stavu se hledá konfigurační soubor \emph{crawler.properties} v~adresáři spuštění aplikace. Cesta ke konfiguračnímu souboru se může změnit přidáním parametru \emph{config}. Výsledný příkaz pro spuštění pak vypadá například následovně: \\ \verb|run.sh -Dconfig=cesta/k/souboru/nastavení.properties| \\

Konfigurační soubor má formu textového souboru, kde každý parametr je na novém řádku. Parametry jsou ve tvaru \verb|proměnná=hodnota|. Chcete-li měnit chování programu pomocí vstupních parametrů, musíte je vložit za jméno skriptu \emph{run.sh}, každý uvozený znaky \verb|-D|, a~jednotlivé parametry oddělit mezerou. Příklad najdete níže.
 
Chcete-li provádět indexování adresářů v~pravidelných intervalech, musíte si pomocí nástrojů vašeho operačního systému nastavit spuštění skriptu i~s konfiguračními parametry. 

Základní spuštění pak vypadá takto:
\begin{verbatim}
run.sh -Dsolr.host=http://localhost:8080/solr/collection1 
 -Dindexer.crawler.file_system.paths=~/documents,~/files
\end{verbatim}
Parametry určují adresu solr serveru a~kolekce, ke které se má připojit a~seznam adresářů, které má indexovat. Další volby lze nastavit přidáním dalších parametrů.

\subsection{Vypnutí fáze}
Pro vypnutí \emph{cleaning} fáze aktivujte parametr \verb|cleaner.crawler=none|. Pro vypnutí \emph{indexing} fáze parametr \verb|indexer.crawler=none|.

\subsection{Parsování dokumentů}
Lze specifikovat, jakým způsobem se budou získávat data z~nalezených souborů. Jsou dvě možnosti. Výchozí je použití \emph{txt}, které načítá všechny soubory jako UTF-8 textové soubory. Přidáním parametru \\\verb|indexer.factory.txt.charset=windows-1250| lze nastavit kódování \\ \emph{windows1250} pro načítání souborů. Obdobným způsobem lze zvolit i~jiné kódování.

Druhou možností je použití \emph{Tika} knihovny, která umí získat text i~ze složitějších formátů dokumentů, jako jsou \emph{DOC}, \emph{RTF} a~podobně. Knihovna sama rozpozná typ a~kódování souborů. Tuto možnost aktivujete pomocí parametru \verb|indexer.factory=tika|.

\subsection{Úplné vyčištění indexu}
Přidáním parametru \verb|cleaner.filter=all| se ve fázi čištění nebude kontrolovat existence souborů, ale odstraní se všechny dokumenty z~indexu.

\section{Frontend}
Frontend je k~dispozici hned po spuštění searcher serveru. Připojíte se k~němu pomocí internetového prohlížeče na nastavené adrese \\ \verb|http://<ip adresa serveru>:<port>/|. Ve výchozím stavu je port nastaven na \emph{8080}.